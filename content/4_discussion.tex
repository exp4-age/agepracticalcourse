\section{Zusammenfassung und Diskussion}
Aus den durchgeführten Messungen haben sich die folgenden Größen für das Torsionspendel ergeben:
\begin{equation*}
    D = \qty{2,92(9)e-4}{\newton\meter}
\end{equation*}
\begin{equation*}
    G = \qty{7.14(117)e10}{\newton\per\square\meter}
\end{equation*}
\begin{equation*}
    J = \qty{6,24(20)e-3}{\kilo\gram\meter\squared}
\end{equation*}
Einen Vergleichswert aus der Literaur gibt es auf Grund des spezifischen Aufbaus des Experiments nur für das Torsionsmodul des Stahls. So findet sich in \cite{Walcher2004}: "`$\alpha$-Eisen: $G = \qty{0,84e4}{kp\per\milli\meter\squared}$"'. Dies entspricht einem Wert von $G = \qty{7.46e10}{\newton\per\square\meter}$, was innerhalb einer Standardabweichung des aus den Messungen bestimmten Wertes liegt.

Die relativ hohe Fehlertoleranz des experimentell bestimmten Torsionsmoduls ist in der Hauptsache durch die Genauigkeit der Messung des Drahtdurchmessers verursacht.
Für eine präzisere Bestimmung des Torsionsmoduls mit dem verwendeten Pendel, müsste der Durchmesser mit einer genaueren Mikrometerschraube gemessen werden.

Hierbei kann es zu Fehlern in der Bestimmung gekommen sein, da keine Übung im Umgang mit einer Mikrometerschraube vorhanden war. Somit ist es möglich, dass der ausgeübte Druck beim Einstellen der Mikrometerschraube am Draht in geringem Maße die Dicke des Drahtes verändert hat. Um die Genauigkeit des Experiments zu verbessern, wäre es somit notwendig, eine genauere Bestimmung der Drahtdicke zu erreichen. \\
Weitere statistische Fehler könnten an mehreren Punkten während der Durchführung passiert sein. Einerseits wurde die schon beschriebene Pendelbewegung um die horizontale Achse beobachtet, deren Einfluss auf das Experiment nicht bekannt ist. Andererseits kann es durch gekippte Fenster und eine geöffnete Tür als möglich erachtet werden, dass ein kontinuierlicher Luftstrom einen Einfluss auf die Pendelbewegung hatte.\\
Außerdem wäre zu empfehlen, die Abstände der Gewichte von der Drehachse mit einem genauerem Messwerkzeug zu bestimmen.
