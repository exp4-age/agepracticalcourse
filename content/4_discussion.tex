\section{Zusammenfassung und Diskussion}
Aus den durchgeführten Messungen haben sich die folgenden Größen für das Torsionspendel ergeben:
\begin{equation*}
    D = \qty{2,92(10)e-4}{\newton\meter}
\end{equation*}
\begin{equation*}
    G = \qty{7.1(12)e10}{\newton\per\square\meter}
\end{equation*}
\begin{equation*}
    J = \qty{6,24(21)e-3}{\kilo\gram\meter\squared}
\end{equation*}
Einen Vergleichswert aus der Literaur gibt es auf Grund des spezifischen Aufbaus des Experiments nur für das Torsionsmodul des Stahls. So findet sich in \cite{Walcher2004}: "`$\alpha$-Eisen: $G = \qty{0,84e4}{kp\per\milli\meter\squared}$"'. Dies entspricht einem Wert von $G = \qty{7.46e10}{\newton\per\square\meter}$, was innerhalb einer Standardabweichung des aus den Messungen bestimmten Wertes liegt.

Die relativ hohe Fehlertoleranz des experimentell bestimmten Torsionsmoduls ist in der Hauptsache durch die Genauigkeit der Messung des Drahtdurchmessers verursacht.
Für eine präzisere Bestimmung des Torsionsmoduls mit dem verwendeten Pendel, müsste der Durchmesser mit einer genaueren Mikrometerschraube gemessen werden.
